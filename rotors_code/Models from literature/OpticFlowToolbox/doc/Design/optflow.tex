\documentclass[a4paper]{article}

\begin{document}

\title{OptFlow:\\Matlab-Tools for the calculation of optic flow from
  geometry and trajectory}
\author{Jens Peter Lindemann}
\maketitle

\section{Overview}

Given the geometry of an environment (3D-model) and a trajectory
(3D-positions and gaze directions), calculate for each step of the
trajectory the optic flow sampled at given positions on the surface of
a unit sphere.

\subsection{Algorithm}

\subsubsection{Input}

\begin{itemize}
\item 3D-model of the environment defined by
  \begin{itemize}
  \item a list of 3D-coordinates (vertices)
  \item a list of polygons: For each polygon a list of references to
    vertices
  \end{itemize}

\item the trajectory defined by
  \begin{itemize}  
  \item 3D-coordinates of the current position
  \item three angles for the current gaze direction: yaw, pitch, roll
  \end{itemize}

\item angular coordinates of the sampling positions on the unit sphere
  for the calculation of the optic flow

\end{itemize}

\subsection{Output}

The optic flow sampled at the given sampling points for each step in
the trajectory.

\subsection{Calculation path}

\begin{enumerate}
\item translation and rotation of the 3D-model according to the
  trajectory

\item calculation of environmental distance for each sampling point:
  \begin{itemize}
  \item intersection of all polygon planes with sampling ray
  \item test of intersections: Intersection point inside of the
    polygon?
  \item distance for this sampling point: Minimum of all valid
    intersections
  \end{itemize}

\item calculation of the translational and rotational velocities by
  differentiation of the trajectory

\item calculation of the optic flow using the Koenderink equation
\end{enumerate}

\section{Data structures}

\subsection{3D-Model}

Minimum requirement for the model is the enumeration of the polygons
forming the enviroment. Thus the minimal representation would be two
lists: A list of vertices and a list of polygons formed by these.

The most general approach to represent a 3D model is to represent a
scene graph, a standard data structure used in 3D modelling. However,
this would make the handling of models very complex.

We start with an intermediate level of representation: A model is
constructed from a list of objects. These objects can be single
polygons or complex structures gouped as the user needs them: Objects
are the level of abstraction that can be referenced for visualisation
purposes. 

Example:

The model consists of a cylindrical arena and three cylindrical
objects. The wall and the objects are separately represented as
objects. This allows to draw the flow originating from the wall in one
color, the flow originating from object movement in a different color.

\subsubsection{Matlab data structure}

An object is represented by a Matlab struct:

\begin{tabular}{ll}
obj.vert & a $N \times 3$ matrix containing the N vertices of the
object.\\ 

obj.tri & a $M \times 3$ matrix enumerating M triangles via indices of
obj.vert\\

obj.poly & a cell of index vectors for the polygons\\

obj.label & a string label for object reference
\end{tabular}

Why represent triangles and polygons? For several reasons it is more
conventient for the flow caluclation to work with triangles. Every
polgon can be represented by a set of triangles. But for display
purposes it may be confusing to draw triangles where the user
specified a square or a hexagon. Thus we store the original polygon
for displaying alongside the polygon split in triangles for
calculation.

Object structures are collected in a Matlab cell array to represent a
model.

Example:

\begin{verbatim}
model={struct('vert', [0,0,0; 1,0,0; 0,1,0; 1,1,0],...
              'tri', [1,2,3; 3,2,4],...
              'poly', {[1,2,3,4]},...
              'label', 'Hsquare'),...
       struct('vert', [0,0,0; 0,1,0; 0,0,1; 0,1,1],...
              'tri', [1,2,3; 3,2,4],...
              'poly', {[1,2,3,4]},...
              'label', 'Vsquare')}
\end{verbatim}

\subsection{Trajectory}

Matlab matrix (double): $T \times 6$ for $T$ trajectory steps.

Column order: x, y, z, yaw, pitch, roll.\newline
x, y, z given in an righthanded cartesian coordinate system.
Yaw pitch and roll describe the rightwise rotation around the axis in the folowing order: \newline
\textbf{yaw} around the z- axis,\newline
\textbf{pitch} around the rotated y- axis and\newline
\textbf{roll} around the rotated x- axis.\newline
All angles given in radian.

Example:
An trayectory of [1, 1, 1, 0, 0, 0] would be the position [1, 1, 1] with viewing-direction along x- axis
and direction of z- axis as top.


\subsection{Sampling points}

Matlab matrix (double): $S \times 2$ for $S$ sampling points.
The sampling points given in spheric coordinates, again all angles in radian:\newline

\textbf{Azimuth} (Longitude) is the angle between x-axis and SamplePoint.
Azimuth angle must be between 0 and $2*\pi$.\newline
[$\pi/2$, 0] = [0, 1, 0] left,\newline
[3*$\pi/2$, 0] = [0, -1, 0] right.


\textbf{Zenith} (Latitude) given in geographic coordinate system is the elevation to the x-y-plane.
Zenith is between $-\pi/2$ and $\pi/2$\newline
[0, $\pi/2$] = [0, 0, 1] top, \newline
[0, $-\pi/2$] = [0, 0, -1] bottom.


\section{Functions}

All functions of the toolbox use prefix ``OF''.

\subsection{Model generation functions}

All functions dealing with models use prefix ``OFMod''

\subsubsection{OFModSphere}

mod=OFModSphere(label, r, density) returns a sphere with radius
r. Optional parameter density specifies density of the mesh.

The sphere forms a single object labeled by the given label.

\subsubsection{OFModRing}

mod=OFModRing(label, r, h, density) returns a circular ring made from
rectangles: r gives the radius of the circle, h the height of the
rectangles. Optional parameter density gives the number of rectangles.

The ring forms a single object labeled by the given label.

\subsubsection{OFModDisc}

mod=OFModDisc(label, r, density) returns a planar circular disc with
radius r. Optional parameter density gives the number of vertices used
to approximate the circle.

The disc forms a single polygon in the object labeled by the given
label.

\subsubsection{OFModCylinder}

mod=OFModCylinder(label, r, h, density) returns a cylinder. See
OFModRing for parameter description. The ring is closed by two Discs.

The cylinder forms a single object labeled by the given label.

\subsubsection{OFModCube}

mod=OFModCube(label, w, h, l) returns a cube with height h, width w
and length l.

The cube forms a single object constructed from rectangles ane labeled
by the given label.

\subsubsection{OFModRectangle}

mod=OFModRectangle(label,w,h) returns a rectangle width width w and
height h.

\subsubsection{OFModPolygon}

mod=OFModPolygon(label, v1, v2, v3, ...) returns a polygon defined by
any number of vertices.

The function tests for planarity of the polygon and gives a warning
otherwise.

\subsubsection{OFObjJoin}

obj=OFObjJoin(label, o1, o2) returns an object constructed from the
components of the two objects o1 and o2. The labels of both input
objects are replaced by the new label given as a parameter.

\subsubsection{OFModJoin}

mod=OFModJoin(mod1, mod2) returns a model composed from two models
mod1 and mod2. This is equivatlent to:

\begin{verbatim}
mod={mod1, mod2}
\end{verbatim}

\subsubsection{OFModTranslate}

mod=OFModTranslate(mod, x, y, z) returns mod moved by translation
vector (x,y,z).

\subsubsection{OFModRotate}

mod=OFModRotate(mod, axis, angle) returns the model rotated by angle
given in radian around the axis given as a three dimensional vector.

\subsubsection{Example}

This small script should create an arena with a cylinder as obstacle
slightly out of center:

\begin{verbatim}
arena=OFModTranslate(OFModRing('wall', 2000, 300), 0, 0, 150);
arena=OFModJoin(arena, OFModDisc('floor', 2000));
arena=OFModJoin(arena, OFModTranslate(OFModCylinder('obstacle', 100, 200, 0);
\end{verbatim}

\subsection{Optic flow calculation functions}

All functions involved in optic flow calculation use prefix ``OFCalc''

\subsubsection{OFCalcDistances}

dist=OFCalcDistances(mod, tra, sp) returns the distances to the
environment at the sampling points for all trajectory steps.  model is
a cell of object structs, tra a $T \times 6$ matrix and sp a $S \times
2$ matrix. Return value dist is a $T \times S$ matrix with environment
distances.
This method is called by OFCalcOpticFlow and dosn't have to be called manually.

\subsubsection{OFCalcOpticFlow}

[hof, vof]=OFCalcOpticFow(mod, tra, sp) returns the optic flow with
model being a cell with object structs, tra a $T \times 6$ matrix and
sp a $S \times 2$ matrix. Return values are both matrices with $T
\times S$ elements where hof(t,s) stored the horizontal and vof(t,s) the
vertical component of the ''optic flow-shifted sampling point'' s at
trajectory position t.

\subsubsection{OFCalcPoE\_PoC}

[PoE, PoC]=OFCalcPoE\_PoC(sp, hof, vof, t) returns the Points which are PoE or PoC, derived by the Optic Flow at position t. PoE and PoC are $n \times 2$ and $m \times 2$ matrices with spheric coordinates (same as sp) of the points.\newline
CAUTION: This works only for SamplePoints generated by OFGenerateSp!
The points can only be found, if their not on the border of the Optic Flow.

\subsubsection{OFCalcVelocities}

[hvel, vvel]=OFCalcVelocities(sp, hof, vof) returns the velocities of the optic flow calculated by the OFCalcOpticFlow method.

\subsubsection{OFGenerateSp}
sp=OFGenerateSp(left, right, top, bottom, density) generates a samplepoint matrix acording to user specifications:\newline
left - the left border of the sp [$pi$ - 0]\newline
right - the right border of the sp [$pi$ - 2*$pi$]\newline
top - the top border of the sp [0 - $pi$/2]\newline
bottom - the bottom border of sp [0 - $pi$/2]\newline
density - the number of sp-rows and columns\newline
Example:
\begin{itemize}
\item OFGenerateSp( $pi$, $pi$, 0, 0, 12) would make 244 samplepoints all at
the front viewing direction (no sence...)

\item OFGenerateSp( $pi$/2, 3*$pi$/2, $pi$/4, -$pi$/4, 11) would make 221 samplepoints
in an area $90\,^{\circ}$ left to $90\,^{\circ}$ right and $45\,^{\circ}$ top to $45\,^{\circ}$ bottom relative to
front viewing direction

\end{itemize}

\subsection{Visualisation functions}

\subsubsection{OFDrawMesh}

OFDrawMesh(mod, show\_tri) draws the mesh defined by mod in the current
figure. If show\_tri is ``True'', the triangles instead of the polygons
are drawn.

\subsubsection{OFDrawTra}

OFDrawTra(mod, tra, mdist) draw the mesh of mod and the trajectory
represented by a line and markers: mdist gives the distance between
markers. Markers are dots on the trajectory and short line segments
illustrating yaw and pitch angles.

\subsubsection{OFDrawCylPlot}

OFDrawCylPlot(sp, hof, vof, t, scale) plots the optic flow field from hof and vof
(see OFCalcOpticFlow) at trajectory position t in an cylindric coordinate system.
Scale scales the Flow Vectorsas follows:\newline
 1=no scale, 0.5= half length, 2= double length\newline
Caution: Flow near the poles seems bigger than it is, due to projection from 
spheric to cylindric coordinate system.

\subsubsection{OFDraw3dPlot}

OFDraw3dPlot(tra, sp, hof, vof, t, scale) plots the optic flow field from hof and vof
(see OFCalcOpticFlow) at trajectory position t on an unitsphere.
Scale scales the Flow Vectors as follows:\newline
 1=no scale, 0.5= half length, 2= double length\newline
If OFDrawTra has been ploted first, this function will insert the unitsphere
with the Optic Flow into the plot at trajectory position t.

\subsection{Subroutine's}

OFSubroutine... are sub-functions called by several other functions.
They do not have to be called manually. 

\end{document}
