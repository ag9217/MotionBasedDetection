\documentclass[a4paper]{article}

\begin{document}

\title{OptFlow:\\Matlab-Tools for the computation of optic flow}
\author{Claudius Strub, Jens Peter Lindemann}
\maketitle

\section{Overview}

This is a short introduction on how to use the OF-toolbox. This toolbox computes the optic flow an agent (e.g. an animal or a robot) experiences when moving in a given environment on a given trajectory. Here the environment is defined by a set of 3D polygons and the trajectory describes a time series of positions and view orientations of the agent. The toolbox computes image velocities resulting from these definitions at sampling points on a spherical visual sensor.

The toolbox functions are grouped by prefixes into the following groups:

\begin{itemize}
\item OFCalc... Computation of optic flow data.
\item OFMod... Creating models.
\item OFDraw... Visualisation of data.
\item OFGenerate... Generate sample points, or stats.
\item OFSubroutine... No need to call these ones. They are used by other Methods.

\end{itemize}

\subsection{You need}

You need three data structures to start
  \begin{enumerate}
  \item a list of 3D-coordinates (vertices) and yaw, pitch, roll angels which describe your Trajectory (tra). This is an $Tx6$ matrix. Angles are in radian!\newline
example: tra=[0 0 0 0 0 0; 0.5 0.5 0 0.15 0 0.2];
  \item a list of spherical coordinates which describe your SamplePoints (sp). This is generated from OFGenerateSp() (see help for parameter details).
  \item a model which describes your world. There are a couple of methods starting with OFMod... which can be used to create one.
  \end{enumerate}

\subsection{Creating a model}
\subsubsection{example}
  world = OFModCylinder('object', 2, 4, 10); \newline
  world = OFModTranslate(world, 4, 2, 0); \newline
  world = OFModJoin(world, OFModCube('EventHorizon', 10, 10, 10));\newline

\subsubsection{infos}
The 3D center of all models is in [0 0 0].\newline
When an model is labelled with 'EventHorizon', it is drawn transparent
(initially used for a surrounding box). All other labelled models will be drawn half-transparent.

\subsubsection{optional: Visualisation} 
 
 \begin{itemize}
  \item Models can be plotted with OFDrawPatch().\newline
example:    OFDrawPatch(world, true);
\item Or a model and trajectory can be plotted together with OFDrawTra(). \newline
example:    OFDrawTra(world, tra, 1);
\end{itemize}

\subsection{compute OpticFlow} 
\subsubsection{example} 
[hof, vof]=OFCalcOpticFlow(world, tra, sp);

\subsubsection{infos} 
To compute the optic flow use OFCalcOpticFlow()
To get good results, trajectory steps should be as small as possible.
(Optic flow is computed with the Koenderink equation \cite{koenderink})
Rotations of more than $pi/4$ between two trajectory steps will lead to inaccurate results. Rotations of more than $pi/2$ will lead to wrong results!

\subsubsection{optional: Visualisation} 
To get the optic flow at trajectory step 2:
  \begin{itemize}
  \item on an 3d sphere: OFDraw3dPlot().\newline
example:    OFDraw3dPlot(tra, sp, hof, vof, 2);
 \item in an cylindrical coordinate system: OFDrawCylPlot().\newline
example:    OFDraw3dPlot(sp, hof, vof, 2);
\end{itemize}
In the cylindric coordinate system the optic flow is shifted, so that the sample point [0 0] is in the middle ($pi$) now. So $pi/2$ is left and $3*pi/2$ is right. 0 and $2*pi$ are backwards.

\begin{thebibliography}{References}
\bibitem{koenderink}J.~J.~Koenderink and Andra J.~van Doorn: Facts on Optic Flow. \textsl{Biological Cybernetics 56(4) pp. 247--254}, 1987
\end{thebibliography}

\end{document}
